
    \documentclass[11pt]{article}

    \usepackage{sectsty}
    \usepackage{graphicx}
    \usepackage{amsmath}
    \usepackage{bm}
    \usepackage{amsfonts}

    % Margins
    \topmargin=-0.45in
    \evensidemargin=0in
    \oddsidemargin=0in
    \textwidth=6.5in
    \textheight=9.0in
    \headsep=0.25in

    \title{ MATH3161 - Optimisation}
    \author{ Generatively made exam by Moae }

    \begin{document}
    \maketitle
    
    \Large \textbf{QUESTION ONE}

    \vspace{5pt}

    \normalsize Consider the matrix $$A =\begin{pmatrix}-6&6\\-4&4\end{pmatrix}.$$ State whether this matrix is positive definite, negative definite, positive semi-definite, negative semi-definite, or indefinite.

    \vspace{20pt}
    
    \Large \textbf{QUESTION TWO}

    \vspace{5pt}

    \normalsize Show that if $A$ is any matrix, then $K = A^T A$ and L = $AA^T$ are both symmetric positive definite matrices.

    \vspace{20pt}
    
    \Large \textbf{QUESTION THREE}

    \vspace{5pt}

    \normalsize If $f$ is a convex function, show that $h(x) = \alpha f(x) + \beta$ is also convex for scalars $\alpha$ and $\beta$ and $\alpha \geq 0$.

    \vspace{20pt}
    
    \Large \textbf{QUESTION FOUR}

    \vspace{5pt}

    \normalsize A construction company is tasked with creating guttering from a strip of metal 12 cm wide. To do this, let $x$ be the length of each of the sections that will form the slope, and let $\theta$ be the angle that each sloping side makes with the horizontal. Determine the values of $x$ and $\theta$ that maximise the capacity of the guttering.

    \vspace{20pt}
    
    \Large \textbf{QUESTION FIVE}

    \vspace{5pt}

    \normalsize Consider the function $f: \mathbb{R}^2\mapsto \mathbb{R}$, where $$f(x_1,~x_2)=7-5x_1x_2-3x_2+10x_2+2x_1.$$ Find and classify all stationary point of this function.

    \vspace{20pt}
    
    \Large \textbf{QUESTION SIX}

    \vspace{5pt}

    \normalsize Consider function $f : \mathbb{R}^2 \mapsto \mathbb{R}$ given by $$f(\boldsymbol{x}) =6x_1^2+7x_2^2+19x_3^2+2x_1x_2+2x_1x_3+4x_2x_3-2x_1+4x_2+4x_3,$$where $\boldsymbol{x} = (x_1, x_2)$ and is subject to the constraint $3x_1+4x_2+4x_3-1=0.$ Now consider the point $\boldsymbol{x}^* = \left[-1, \dfrac{1}{2}, \dfrac{1}{2}\right]^T$. Show that this point is a regular point and that it satisfies the first order necessary conditions. Also verify that $\boldsymbol{x}^*$ satisfies the second order sufficient conditions for a local minimiser of the problem. Is $\boldsymbol{x}^*$ a global minimiser?

    \vspace{20pt}
    
    \Large \textbf{QUESTION SEVEN}

    \vspace{5pt}

    \normalsize Consider the function $f: \mathbb{R}^2\mapsto \mathbb{R}$, where $$f(x_1, x_2)=2x_1^2+4x_2^2-4x_1,$$ subject to the constraint $(x_1 - 1)^2+(x_2 - 1)^2\leq 16$. We aim to minimise this function in the given region defined by the constraints. Sketch the region and confirm whether or not this is a convex optimisation problem. If it is, write down the Kuhn-Tucker conditions for this problem and show that $\boldsymbol{x}^*=\left[1, -3\right]^T$ satisfies them. Is $\boldsymbol{x}^*$ a global minimum?

    \vspace{20pt}
    
    \Large \textbf{QUESTION EIGHT}

    \vspace{5pt}

    \normalsize Consider the function $f: \mathbb{R}^2\mapsto \mathbb{R}$, where $$f(x_1, x_2)=2x_1^2+x_2^2+2x_1x_2+4x_1,$$ subject to the constraints $-3x_1-x_2+2\leq0$ and $x_1 \geq 0$. We aim to minimise this function in the given region defined by the constraints. Write down the (Wolfe) dual problem, if applicable. Otherwise, show that the given problem is not convex.

    \vspace{20pt}
    
    \Large \textbf{QUESTION NINE}

    \vspace{5pt}

    \normalsize Consider the function: $$f(x_1, x_2) = 2x_1^2+3x_2^2+2x_1-2x_2.$$Prove by induction that the method of steepest descent applied with an initial guess $\boldsymbol{x}^{(1)} = \boldsymbol{0}$ generates the sequence $\{\boldsymbol{x}^{(k)}\}$ where $$\{\boldsymbol{x}^{(k + 1)}\} = \left(-\left(\dfrac{1}{5}\right)^k+\dfrac{1}{2},~\left(-\dfrac{1}{5}\right)^k-\dfrac{1}{3}\right).$$ Hence, deduce the minimiser.

    \vspace{20pt}
    
    \Large \textbf{QUESTION TEN}

    \vspace{5pt}

    \normalsize Consider minimising the function $f: \mathbb{R}^2 \mapsto \mathbb{R}$ given by $$f(\boldsymbol{x}) = x_1^2+3x_2^2-x_1-x_2-5$$ starting from $\boldsymbol{x}^{(1)} = \left[-1, 2\right].$ Show that $f$ is a strictly convex function and find its minimiser $\boldsymbol{x}^*$ and $f(\boldsymbol{x}^*)$. Calculate the Newton direction and confirm that it is a descent direction. Finally, use Newton's method to find the minimiser of $f$ starting from $\boldsymbol{x}^{(1)}$. Comment on your results.

    \vspace{20pt}
    
    \Large \textbf{QUESTION ELEVEN}

    \vspace{5pt}

    \normalsize Consider minimising the function $f: \mathbb{R}^2 \mapsto \mathbb{R}$ given by $$f(\boldsymbol{x}) = 3x_1^2+3x_2^2-3x_1x_2+x_1+1$$ starting from $\boldsymbol{x}^{(1)} = \left[0, 2\right].$ Show that $f$ is a strictly convex function and find its minimiser $\boldsymbol{x}^*$ and $f(\boldsymbol{x}^*)$. Use a conjugate gradient method with the Fletcher-Reeves update to minimise $f$ starting from $\boldsymbol{x}^{(1)}$. Verify that the search directions are descent directions and that the search directions are conjugate with respect to the Hessian of $f$.

    \vspace{20pt}
    \end{document}